\documentclass[12pt, a4paper]{article} %minska pointen om du vill ha mindre bokstäver

\usepackage[english]{babel}
\usepackage[utf8]{inputenc} 
\usepackage[T1]{fontenc} 	

\usepackage{amsmath}	% Om du vill använda AMSLaTeX 
\usepackage{amssymb}	% Om du vill använda AMS symboler
\usepackage{amsfonts}
\usepackage{amsthm}
\usepackage{enumerate}
\usepackage[hidelinks]{hyperref}
\usepackage{tikz} % för tikz-figurer
\usepackage{pgfplots}
\usepackage{csvsimple}
\usepackage{booktabs}

\pgfplotsset{width=0.75\textwidth,compat=1.9}
\usepgfplotslibrary{statistics}

\usepackage[parfill]{parskip}

\usepackage{subcaption} % för flera figurer i en

% Färger för exempelkoden
\usepackage{xcolor}
\definecolor{backcol}{gray}{0.95} % bakgrundfärg
\definecolor{darkpastelgreen}{rgb}{0.01, 0.75, 0.24} % bl.a. kommentarer i koden

\usepackage{listings} % för exempelkod
% Inställningar
\lstset{
	basicstyle=\ttfamily\scriptsize, % monospace, mindre bokstäver
	basewidth  = {.5em,0.5em}, % minskar avståndet mellan bokstäverna (passar fonten)
	numbers=left, numberstyle=\tiny, 
	frame=tb, % streck top och bottom
	breaklines=true, % radbyte vid behov
	backgroundcolor = \color{backcol},
	keywordstyle=\color{blue}, 
	commentstyle=\color{darkpastelgreen},
	captionpos=t,
	framexleftmargin=2mm, % padding
	xleftmargin=2mm, % lägg till marginal för att hållas i linje med texten
	framexrightmargin=2mm,
	xrightmargin=2mm,
	}

\title{Operating systems -- Assignment 2\\Scheduling}

\author{Lennart Jern\\
	CS: ens16ljn\\ \\ \textbf{Teacher}\\ Ahmed Aley}


\begin{document}


\maketitle

\newpage



\begin{figure}
	\begin{tikzpicture}
	\begin{axis}[
	ymin=0,
	xlabel=Number of threads,
	ylabel=Time (s)
	]
	\addplot table [x=Threads, y=Normal, col sep=comma] {data/thread_max.csv};
	\addplot table [x=Threads, y=Batch, col sep=comma] {data/thread_max.csv};
	\addplot table [x=Threads, y=Idle, col sep=comma] {data/thread_max.csv};
	\addplot table [x=Threads, y=FIFO, col sep=comma] {data/thread_max.csv};
	\addplot table [x=Threads, y=Round Robin, col sep=comma] {data/thread_max.csv};
	\legend{FIFO,Round Robin}
	\end{axis}
	\end{tikzpicture}
	\caption{Threads max.}
	\label{threads-max}
\end{figure}

\begin{figure}
	\begin{tikzpicture}
	\begin{axis}[
	ymin=0,
	xlabel=Number of threads,
	ylabel=Time (s)
	]
	\addplot table [x=Threads, y=Normal, col sep=comma] {data/thread_min.csv};
	\addplot table [x=Threads, y=Batch, col sep=comma] {data/thread_min.csv};
	\addplot table [x=Threads, y=Idle, col sep=comma] {data/thread_min.csv};
	\addplot table [x=Threads, y=FIFO, col sep=comma] {data/thread_min.csv};
	\addplot table [x=Threads, y=Round Robin, col sep=comma] {data/thread_min.csv};
	\legend{FIFO,Round Robin}
	\end{axis}
	\end{tikzpicture}
	\caption{Threads min.}
	\label{threads-min}
\end{figure}


\section{Introduction}

The Linux kernel provides a number of different scheduling policies that can be used to fine tune the performance of certain applications.
In this report, five different schedulers are evaluated using an artificial, CPU intensive, work load.
Three of the tested schedulers are ``normal'', while the last two are ``real-time'' schedulers, meaning that they provide higher priority for their processes than the normal ones do.

The work load consists of a simple program, called \texttt{work}, that sums over a part of Grandi's series\footnote{\url{https://en.wikipedia.org/wiki/Grandi's_series}} ($1-1+1-1+\dots$), using a specified number of threads.
Since the task is easy to parallelize, only require minimal memory access and no disk access, it should be comparable to CPU intense tasks like compression and matrix calculations.

\section{Implementation (Method?)}

A Bash script (\texttt{timer.sh}) was used to collect data by timing the work load 10 times for each scheduler, for thread counts ranging from 1 to 10.
See code listing \ref{timer} for the code.
The data was then processed by a simple Python program in order to calculate the median, minimum and maximum run time for each scheduler and thread count.
It should be noted here that the real-time schedulers were run with maximum priority.
The other schedulers does not accept any priority settings.

All tests were run on my personal computer with the specifications seen in table \ref{spec}.

% Versions of relevant packages, makefile
% Pandas, bash, python, gcc

\begin{table}[h]
	\centering
	\begin{tabular}{ll}
		Component & Specification \\
		\hline
		OS: & Fedora 25 \\
		Kernel: & Linux 4.8.12-300.fc25.x86\_64 \\
		CPU: & Intel Core i5-2500K CPU @ 3.7GHz \\
		RAM: & 7965MiB
	\end{tabular}
	\caption{Test system specification}
	\label{spec}
\end{table}



\section{Results}

An immediate inspection of the timing data does not reveal any significant differences between the schedulers, not even between the normal and real-time ones.
The median run times can be seen in figure \ref{median} for the normal schedulers and in figure \ref{median-real} for the real-time schedulers.
Similarly, the maximum and minimum run times for the normal, batch and idle schedulers can be seen in figures \ref{max} and \ref{min} respectively, while the real-time equivalents appears in figures \ref{max-real} and \ref{min-real}.

It is more interesting to compare the range of response times between the schedulers (fig. \ref{range}).
This reveals a clear difference between the real-time and normal schedulers, where the real-time ones are clearly more predictable for two or more threads.

The raw data collected can be found in appendix \ref{raw}.


\begin{figure}
	\centering
	\begin{tikzpicture}
	\begin{axis}[
	ymin=0,
	xlabel=Number of threads,
	ylabel=Time (s)
	]
	\addplot table [x=Threads, y=Normal, col sep=comma] {data/thread_medians.csv};
	\addplot table [x=Threads, y=Batch, col sep=comma] {data/thread_medians.csv};
	\addplot table [x=Threads, y=Idle, col sep=comma] {data/thread_medians.csv};
	\addplot table [x=Threads, y=FIFO, col sep=comma] {data/thread_medians.csv};
	\addplot table [x=Threads, y=Round Robin, col sep=comma] {data/thread_medians.csv};
	\legend{Normal,Batch,Idle,FIFO,Round Robin}
	\end{axis}
	\end{tikzpicture}
	\caption{The median time per thread.}
	\label{threads-medians}
\end{figure}

\begin{figure}[ht]
	\centering
	\begin{tikzpicture}
		\begin{axis}[
		width=0.75\textwidth,
		ymin=0,
		xlabel=Number of threads,
		ylabel=Time (s),
		legend cell align=left,
		]
		\addplot table [x=Threads, y=Normal, col sep=comma] {data/medians.csv};
		\addplot table [x=Threads, y=Batch, col sep=comma] {data/medians.csv};
		\addplot table [x=Threads, y=Idle, col sep=comma] {data/medians.csv};
		\addplot table [x=Threads, y=FIFO, col sep=comma] {data/medians.csv};
		\addplot table [x=Threads, y=Round Robin, col sep=comma] {data/medians.csv};
		\legend{Normal,Batch,Idle,FIFO,Round Robin}
		\end{axis}
	\end{tikzpicture}
	\caption{The median time required to finish the complete task.}
	\label{median}
\end{figure}

\begin{figure}
	\centering
	\begin{subfigure}[t]{0.45\textwidth}
		\centering
		\begin{tikzpicture}
		\begin{axis}[
		width=\textwidth,
		ymin=0,
		xlabel=Number of threads,
		ylabel=Time (s)
		]
		\addplot table [x=Threads, y=Normal, col sep=comma] {../data/range.csv};
		\addplot table [x=Threads, y=Batch, col sep=comma] {../data/range.csv};
		\addplot table [x=Threads, y=Idle, col sep=comma] {../data/range.csv};
		\legend{Normal,Batch,Idle}
		\end{axis}
		\end{tikzpicture}
		\caption{The range for normal schedulers}
		\label{spread}
	\end{subfigure}
	~
	\begin{subfigure}[t]{0.45\textwidth}
		\centering
		\begin{tikzpicture}
		\begin{axis}[
		width=\textwidth,
		ymin=0,
		xlabel=Number of threads,
		ylabel=Time (s)
		]
		\addplot table [x=Threads, y=FIFO, col sep=comma] {../data/range.csv};
		\addplot table [x=Threads, y=RR, col sep=comma] {../data/range.csv};
		\legend{FIFO,Round Robin}
		\end{axis}
		\end{tikzpicture}
		\caption{The range for real time schedulers.}
		\label{spread-real}
	\end{subfigure}
	\caption{A comparison of response times for the complete task, between the different schedulers.}
	\label{range}
\end{figure}

\begin{figure}
	\centering
	\begin{subfigure}[t]{0.48\textwidth}
		\includegraphics[width=\textwidth]{figures/box1.pdf}
	\end{subfigure}
	~
	\begin{subfigure}[t]{0.48\textwidth}
		\includegraphics[width=\textwidth]{figures/box2.pdf}
	\end{subfigure}
	
	\begin{subfigure}[t]{0.48\textwidth}
		\includegraphics[width=\textwidth]{figures/box3.pdf}
	\end{subfigure}
	~
	\begin{subfigure}[t]{0.48\textwidth}
		\includegraphics[width=\textwidth]{figures/box4.pdf}
	\end{subfigure}
	
	\begin{subfigure}[t]{0.48\textwidth}
		\includegraphics[width=\textwidth]{figures/box5.pdf}
	\end{subfigure}
	~
	\begin{subfigure}[t]{0.48\textwidth}
		\includegraphics[width=\textwidth]{figures/box6.pdf}
	\end{subfigure}
	\caption{box1}
	\label{}
\end{figure}

\begin{figure}
	\centering
	\begin{subfigure}[t]{0.48\textwidth}
		\includegraphics[width=\textwidth]{figures/box7.pdf}
	\end{subfigure}
	~
	\begin{subfigure}[t]{0.48\textwidth}
		\includegraphics[width=\textwidth]{figures/box8.pdf}
	\end{subfigure}
	
	\begin{subfigure}[t]{0.48\textwidth}
		\includegraphics[width=\textwidth]{figures/box9.pdf}
	\end{subfigure}
	~
	\begin{subfigure}[t]{0.48\textwidth}
		\includegraphics[width=\textwidth]{figures/box10.pdf}
	\end{subfigure}
	\caption{box2}
	\label{}
\end{figure}

\begin{figure}
	\centering
	\begin{subfigure}[t]{0.9\textwidth}
		\includegraphics[width=\textwidth]{figures/density2.pdf}
		\caption{Thread count: 2}
	\end{subfigure}

	\begin{subfigure}[t]{0.9\textwidth}
		\includegraphics[width=\textwidth]{figures/density4.pdf}
		\caption{Thread count: 4}
	\end{subfigure}
	
	\caption{Density plots for the different schedulers when running with 2 and 4 thread}
	\label{density1}
\end{figure}

\begin{figure}
	\centering
	\begin{subfigure}[t]{0.9\textwidth}
		\includegraphics[width=\textwidth]{figures/density6.pdf}
		\caption{Thread count: 6}
		\label{dens6}
	\end{subfigure}

	\begin{subfigure}[t]{0.9\textwidth}
		\includegraphics[width=\textwidth]{figures/density8.pdf}
		\caption{Thread count: 8}
		\label{dens8}
	\end{subfigure}
	\caption{Density plots for the different schedulers when running with 6 and 8 thread}
	\label{density2}
\end{figure}

\section{Final thoughts and lessons learned}

\pagebreak
\appendix

\section{Code listings}

\lstinputlisting[language=c, caption=work.c, label=work]{../code/work.c}

\lstinputlisting[language=bash, caption=timer.sh, label=timer]{../code/timer.sh}

\lstinputlisting[language=python, caption=stats.py, label=stats]{../code/stats.py}

\lstinputlisting[caption=Makefile, label=make]{../code/Makefile}

\section{Raw data}\label{raw}

\csvautobooktabular{data/medians.csv}

\csvautobooktabular{data/max.csv}

\csvautobooktabular{data/min.csv}

%\csvautobooktabular{data/data1.csv}
%\csvautobooktabular{data/data2.csv}
%\csvautobooktabular{data/data3.csv}
%\csvautobooktabular{data/data4.csv}
%\csvautobooktabular{data/data5.csv}
%\csvautobooktabular{data/data6.csv}
%\csvautobooktabular{data/data7.csv}
%\csvautobooktabular{data/data8.csv}
%\csvautobooktabular{data/data9.csv}
%\csvautobooktabular{data/data10.csv}

\end{document}
